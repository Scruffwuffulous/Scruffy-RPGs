\section{Characters and how to make them}

\subsection{Roles}

Roles describe how well your character fits into a certain role; how well they act at wearing the particular hats that are involved in their day-to-day.

Roles are split into different \textbf{Role Classes}\index{Role Classes}. These are as follows:
\begin{enumerate}
    \item Investigator\index{Investigator Roles}: How you deal with an unknown situation
    \begin{enumerate}
        \item Detective\index{Detective}: A keen eye for detail and discrepancy; with enough focus and deductive logic, nothing will remain hidden for long.
        \item Journalist\index{Journalist}: With a open ear and an eye for sources, the journalists know how to get to the heart of a scoop for better or worse
        \item Sheriff\index{Sheriff}: You are the kind of person to knock down every door to get your man. Your jurisdiction has no boundaries, and neither do your methods.
        \item Secret Fourth Thing!
    \end{enumerate}
    \item Social\index{Social Roles}: How you interact with other people
    \begin{enumerate}
        \item Team Player\index{Team Player}: You feel at ease only when you know you are part of a larger whole
        \item Natural Leader\index{Natural Leader}: You feel like you could take charge in a crisis, or at least manage a half-decent meeting. In any case, you feel more comfortable giving orders than taking them.
        \item Lone Wolf\index{Lone Wolf}: You can't trust anyone else to do the job right, you say. But do you really believe that?
        \item
    \end{enumerate}
    \item Psyche\index{Psyche Roles}: How you cope with the noise inside your head
    \begin{enumerate}

        \item Escapist\index{Escapist}: Full of facts and figures, or full of substances and feelings, just so long as you are not full of thoughts that you did not call forth
        \item Therapist\index{Therapist}: Over-analysis of every thought has led you to what you think is understanding of how you and everyone you meet works.
        \item Officer\index{Officer}: A strict routine and set of habits prevent you from being too close to your thoughts and problems. They can't hurt you as long as you follow the rules, right?
        \item Cynic\index{Cynic}: You've given in to the negative thoughts of life, your friends, and the world at large. At least you don't have to fight it any more.
    \end{enumerate}
\end{enumerate}

\subsubsection{Investigator Roles}

The Investigator roles are how your character interacts with unknown situations and information gathering. This will be a large part of the game as you find bizarre scenes of crimes and conspiracy and will have to get to the truth. The question you must ask yourself is why must you find this truth?

The \textbf{Detective}\index{Detective} enjoys the mystery for its own sake, the intricacies of the puzzle hold their own beauty. If you have a high dice rating in this role, then your character is capable of analysing a clue, object, or situation with clinical precision. This may be from a natural gift of insight, or the careful application of a process, or simply a strong gut that points them towards the important aspects of what they see.

Examples of characters with high Detective ratings would be:
\begin{itemize}
    \item Sherlock Holmes
    \item
\end{itemize}

Be aware that having a high detective rating does \textbf{not} mean that your character is necessarily a detective by trade, or even wants to engage in this life. It only means that their skillset fits this role well.

\vspace{12pt}

The \textbf{Journalist}\index{Journalist}
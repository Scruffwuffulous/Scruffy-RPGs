\section{Characters and how to make them}

\begin{ChapterSummary}
    \item What makes up a character?
    \begin{ChapterSummary}
        \item Labels
        \item The F.I.R.S.T. system
        \item Relationships
        \item Personal Items
        \item Powers
    \end{ChapterSummary}
    \item How to build your own character?
    \begin{ChapterSummary}
        \item The standard method
        \item Some example characters
        \item Things you can tweak (and might want to)
    \end{ChapterSummary}
    \item Non-Player Characters and the Supporting Cast
    \begin{ChapterSummary}
        \item Supporting Characters
        \item GMPCs \& NPCs
    \end{ChapterSummary}
\end{ChapterSummary}
\pagebreak

\subsection{What makes up a character?}

In this chapter, we shall go over the character \textbf{stats} and what they represent, both in terms of a character's physical and mental attributes, and their personality, traits, and other characteristics. If you have played a TTRPG before, like \textbf{Dungeons \& Dragons}, some of these will be somewhat familiar to you. Others take more inspiration from less crunchy games, especially those that are \textbf{Powered by the Apocalypse}. If this seems like nonsense to you, don't Worry! Hopefully all will be explained in this chapter...

To start we are going to go through the various \textbf{Attributes}\index{Attributes} that your character will have in their categories. These serve two functions: to represent how proficient, skilled, or otherwise capable a character is in a given area, and to tell you how many and what kind of \textbf{DICE} to roll when the GM asks you to. This second function will be discussed later in the chapter on mechanics, but we will give a brief overview in each subsection as to when you may be required to roll with a given attribute.

If you are used to RPGs, you may be accustomed to modifiers and arbitrary values associated with

\subsubsection{Labels}

The first, and most important attribute we will learn about, is a character's \textbf{Labels}\index{Labels}. Each character has 3 Labels. Each of these is simply a sentence or phrase that describes an aspect of your character. These are tricky to describe, so we shall start with some examples:

Recall everyone's favourite Grey Wizard, Gandalf. He may have some Labels like:
\begin{itemize}
    \item Wielder of the flame of Anor
    \item Maiar Wizard of the Istari Order
\end{itemize}
But these only seem to reflect the fact that he has great magical power. Note that the first one emphasises that he "wields" the flame, as opposed to "conquering" or "fumbling". In the second the fact that he is a member of the order tells us that he may have some sway with people who respect that order. Be careful with Labels, as words have power here.

A few more for Gandalf:
\begin{itemize}
    \item Quick to anger, quicker to laugh
    \item Wise sage guide of the fellowship
    \item Ancient friend to elves, men, and dwarves alike
\end{itemize}
Ask yourself how much we can learn about Gandalf from these phrases, and whether you agree that this is accurate to how you see his character. If you want more practice try taking your favourite characters and try to boil them down to a few sentences. Once you can get three, you are ready.

\subsubsection{The \First system}

The \First system is the next set of character attributes worth covering

Roles describe how well your character fits into a certain role; how well they act at wearing the particular hats that are involved in their day-to-day.

Roles are split into different \textbf{Role Classes}\index{Role Classes}. These are as follows:
\begin{itemize}
    \item Fighter\index{Fighter}
    \item Investigator\index{Investigator}
    \item Rebel\index{Rebel}
    \item Street Punk\index{Street Punk}
    \item Technomancer\index{Technomancer}
\end{itemize}

\subsubsection{Relationships}

\subsubsection{Personal Items}

\subsubsection{Powers}


\pagebreak
\subsection{How to build your own character?}

\subsubsection{The standard method}

\subsubsection{Some example characters}

\subsubsection{Things you can tweak (and might want to)}


\pagebreak
\subsection{Non-Player Characters and the Supporting Cast}

\subsubsection{Supporting Characters}

\subsubsection{GMPCs \& NPCs}
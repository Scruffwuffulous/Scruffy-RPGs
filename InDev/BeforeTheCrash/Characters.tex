\section{Characters and how to make them}

\begin{ChapterSummary}
    \item What makes up a character?
    \begin{ChapterSummary}
        \item Labels
        \item The F.I.R.S.T. system
        \item Relationships
        \item Personal Items
        \item Powers
        \item Vices
    \end{ChapterSummary}
    \item How to build your own character?
    \begin{ChapterSummary}
        \item The standard method
        \item Some example characters
        \item Things you can tweak (and might want to)
    \end{ChapterSummary}
    \item Non-Player Characters and the Supporting Cast
    \begin{ChapterSummary}
        \item Supporting Characters
        \item GMPCs \& NPCs
    \end{ChapterSummary}
\end{ChapterSummary}
\pagebreak

\subsection{What makes up a character?}

In this chapter, we shall go over the character \textbf{stats} and what they represent, both in terms of a character's physical and mental attributes, and their personality, traits, and other characteristics. If you have played a TTRPG before, like \textbf{Dungeons \& Dragons}, some of these will be somewhat familiar to you. Others take more inspiration from less crunchy games, especially those that are \textbf{Powered by the Apocalypse}. If this seems like nonsense to you, don't Worry! Hopefully all will be explained in this chapter...

To start we are going to go through the various \textbf{Attributes}\index{Attributes} that your character will have in their categories. These serve two functions: to represent how proficient, skilled, or otherwise capable a character is in a given area, and to tell you how many and what kind of \textbf{DICE} to roll when the GM asks you to. This second function will be discussed later in the chapter on mechanics, but we will give a brief overview in each subsection as to when you may be required to roll with a given attribute.

If you are used to RPGs, you may be accustomed to modifiers and arbitrary values associated with proficiency or lack thereof. Here, we only deal in that greatest of objects: \textbf{DICE}! Attributes (as well as some other things we shall see in the next chapter) are each rated with dice, with the lowest a \dfour and the highest a \dtwelve In most cases, a \dfour represents below average skill or capability, and \dtwelve represents extreme or otherwise incredible capability. Logically, you may have guessed that \deight sits around average, and if your GM is being kind, it should.

\subsubsection{Labels}

The first, and most important attribute we will learn about, is a character's \textbf{Labels}\index{Labels}. Each character has 3 Labels. Each of these is simply a sentence or phrase that describes an aspect of your character. These are tricky to describe, so we shall start with some examples:

Recall everyone's favourite Grey Wizard, Gandalf. He may have some Labels like:
\begin{itemize}
    \item Wielder of the flame of Anor \deight
    \item Maiar Wizard of the Istari Order \deight
\end{itemize}
But these only seem to reflect the fact that he has great magical power. Note that the first one emphasises that he "wields" the flame, as opposed to "conquering" or "fumbling". In the second the fact that he is a member of the order tells us that he may have some sway with people who respect that order. Be careful with Labels, as words have power here.

A few more for Gandalf:
\begin{itemize}
    \item Quick to anger, quicker to laugh \deight
    \item Wise sage guide of the fellowship \deight
    \item Ancient friend to elves, men, and dwarves alike \deight
\end{itemize}
Ask yourself how much we can learn about Gandalf from these phrases, and whether you agree that this is accurate to how you see his character. If you want more practice try taking your favourite characters and try to boil them down to a few sentences. Once you can get three, you are ready.

When assigning dice values to Labels, you will always give them a \deight. This may seem odd: "How can I be average at being me?" is a perfectly reasonable question to ask in this instance. In practice this reflects that whenever your character is acting in line with who they basically are, they should bring some of their core competencies into action. Again, more on when dice matter in the next chapter...

\subsubsection{The \First system}

The \First system is the next set of character attributes worth covering.

Roles describe how well your character fits into a certain role; how well they act at wearing the particular hats that are involved in their day-to-day. These should cover situations when the outcome is uncertain. Maybe not how your character would act browsing a supermarket, more how they would act when confronted by somebody they don't know being too familiar or strongly political with them in a supermarket.

Roles are split into different \textbf{Role Classes}\index{Role Classes}. These are as follows:
\begin{itemize}
    \item Fighter\index{Fighter}
    \item Investigator\index{Investigator}
    \item Rebel\index{Rebel}
    \item Street Punk\index{Street Punk}
    \item Technomancer\index{Technomancer}
\end{itemize}
Hopefully, most of these should be somewhat straightforward, but we'll go through each individually shortly. Before that, I should mention that when making your character you will also get to pick specialities within these roles. Specialities\index{Specialities} are single word modifiers that represent the style in which your character acts with some of their more familiar roles. Hopefully this will become clear as you read on...

\textbf{FIGHTER:}\index{Fighter}

To help explain, we start with the first, and most straight-forward of the roles. A fighter is someone who is skilled at or otherwise accustomed to fighting (who would have guessed!). This encompasses a range of traits. It could mean anything from being outwardly violent and ready to fight with fists or words at the drop of a hat. Or it could describe a skilled martial artist who does not rise to the challenge, but is capable at responding to it.

A character with a high rating in \textbf{Fighter} is quite capable and/or knowledgeable about combat, and would be good at provoking, baiting, or wounding others with their words. A character with a low \textbf{Fighter} rating however, would not last long in a straight fight, and probably doesn't have the charisma required to stare someone down through sheer toughness.

With this in mind, some example specialities for this might be:
\begin{multicols}{2}
\begin{itemize}
    \item Boxer
    \item Belligerent
    \item Fencer
    \item Grappler
    \item Scrapper
    \item Sniper
\end{itemize}
\end{multicols}
although this is by \textit{no means exhaustive} and if you come up with a modifier that you think fits this role, give it a try!

\textbf{INVESTIGATOR}\index{Investigator}

Next up, the Investigator. An investigator is someone who is skilled at analysing an unknown situation. Whether this is a crime scene, a crowd that has formed in the street, or just a dangling thread in conversation, investigators are the people who will hunt down the answers

A character with a high \textbf{Investigator} rating is likely to get to the bottom of any puzzle or clue that crosses their path, at least eventually. On the other hand, a low \textbf{Investigator} rating would suggest that this character is not the most observant or that they prefer to find their answers through less analytical means.

Once again, we have some example specialities for this:
\begin{multicols}{2}
\begin{itemize}
    \item Detective
    \item Journalist
    \item Sheriff
    \item
\end{itemize}
\end{multicols}
(You can probably match each of these to a different fictional detective if you're trying hard enough)

\textbf{Rebel}\index{Rebel}

Ah, the rebel. Whether your character has a cause or not, this role is for those who stand apart and shine bright doing so. Nobody is happy with the state of the world, but rebels are the ones who are doing something about it. That, or they are the ones yelling over the rest of the noise.

A character with a high \textbf{Rebel} rating might be better informed as to the political goings-on both what's in the news, and what might soon be. Similarly they might be better at rousing support from the people (or having the opposite effect on corpos). On the other hand, a character with a lower \textbf{Rebel} rating might struggle when trying to act against the orders of the corps and the law. They also may find it hard trying to impose their own will on themselves as well, the voices are not always kind.

It's time for some rebellious specialities, I think:
\begin{multicols}{2}
\begin{itemize}
    \item Artist
    \item Criminal
    \item Firebrand
    \item Revolutionary
    \item Rockstar
    \item Terrorist
\end{itemize}
\end{multicols}
Note that these are not all mutually exclusive. It doesn't matter so much that the speciality is unique, only that it matches your vision for the character.

\subsubsection{Relationships}

\subsubsection{Personal Items}

\subsubsection{Powers}

\subsubsection{Vices}


\pagebreak
\subsection{How to build your own character?}

\subsubsection{The standard method}

\subsubsection{Some example characters}

\subsubsection{Things you can tweak (and might want to)}


\pagebreak
\subsection{Non-Player Characters and the Supporting Cast}

\subsubsection{Supporting Characters}

\subsubsection{GMPCs \& NPCs}
